%%%%%%%%%%%%%%%%%%%%%%%%%%%%%%%%%%%%%%%%%%%%%%
%                                            %
%   W Z O R Z E C   S P R A W O Z D A N I A  %
%                                            %
%%%%%%%%%%%%%%%%%%%%%%%%%%%%%%%%%%%%%%%%%%%%%%


\documentclass[12pt,a4paper,oneside]{article}

\usepackage{amsmath,amssymb}
\usepackage[utf8]{inputenc}                                      
\usepackage[OT4]{fontenc}      
%\usepackage[T1]{fontenc}                            
\usepackage[polish]{babel}                           
\selectlanguage{polish}
\usepackage{indentfirst}
\usepackage{fancyvrb} 
\usepackage[dvips]{graphicx}
\usepackage{tabularx}
\usepackage{float}
\usepackage{color}
\usepackage{rotating}
\usepackage{hyperref} 
\usepackage{fancyhdr}
\usepackage{listings}
\usepackage{booktabs}
\usepackage{ifpdf}
\usepackage{mathtext} % polskie znaki w trybie matematycznym
%\makeindex  % utworzenie skorowidza (w dokumencie pdf)
\usepackage{lmodern}
%\usepackage[osf]{libertine}
\usepackage{filecontents}
\usepackage{ifthen}
\usepackage{spverbatim}


\usepackage{tikz}
\usetikzlibrary{arrows}


\newcounter{nextYear}
\setcounter{nextYear}{\the\year}
\stepcounter{nextYear}

% rozszerzenie nieco strony
%\setlength{\topmargin}{-1cm} \setlength{\textheight}{24.5cm}
%\setlength{\textwidth}{17cm} \addtolength{\hoffset}{-1.5cm}
%\setlength{\parindent}{0.5cm} \setlength{\footskip}{2cm}
%\linespread{1.2} % odstep pomiedzy wierszami


%%%% ZYWA PAGINA %%%%%%%%%%%
\newcommand{\tl}[1]{\textbf{#1}} 
\pagestyle{fancy}
\renewcommand{\sectionmark}[1]{\markright{\thesection\ #1}}
\fancyhf{} % usuwanie bieżących ustawień
\fancyhead[L,R]{\small\bfseries\thepage}
\fancyhead[L]{\small\bfseries\rightmark}
\fancyhead[R]{\small\bfseries\leftmark}
\renewcommand{\headrulewidth}{0.5pt}
\renewcommand{\footrulewidth}{0pt}
\addtolength{\headheight}{0.5pt} % pionowy odstęp na kreskę
\fancypagestyle{plain}{%
\fancyhead{} % usuń p. górne na stronach pozbawionych numeracji
\renewcommand{\headrulewidth}{0pt} % pozioma kreska
}

%%%%%   LISTINGI %%%%%%%%
% ustawienia listingu programow

\lstset{%
language=C++,%
commentstyle=\textit,%
identifierstyle=\textsf,%
keywordstyle=\sffamily\bfseries, %
%captionpos=b,%
tabsize=3,%
frame=lines,%
numbers=left,%
numberstyle=\tiny,%
numbersep=5pt,%
breaklines=true,%
morekeywords={pWezel,Wezel,string,ref,params_result},%
escapeinside={(*@}{@*)},%
%basicstyle=\footnotesize,%
%keywords={double,int,for,if,return,vector,matrix,void,public,class,string,%
%float,sizeof,char,FILE,while,do,const}
}
%%%%%%%%%%%%%%%%%%%%%%%%%%%%%%%%%%%%%%%%%%%%%%%%%%%%%%%%%%%%%%%%%%%%%%%

%%%%%%%%%  NOTKI NA MARGINESIE %%%%%%%%%%%%%
% mala zmiana sposobu wyswietlania notek bocznych
\let\oldmarginpar\marginpar
\renewcommand\marginpar[1]{%
  {\linespread{0.85}\normalfont\scriptsize%
\oldmarginpar[\hspace{1cm}\begin{minipage}{3cm}\raggedleft\scriptsize\color{black}\textsf{#1}\end{minipage}]%    left pages
{\hspace{0cm}\begin{minipage}{3cm}\raggedright\scriptsize\color{black}\textsf{#1}\end{minipage}}% right pages
}%
}
% % % % % % % % % % % % % % % % % % % % % % % % % % % % % % % %

%%%% WYSWIETLANIE AKTUALNEGO ROKU AKADEMICKIEGO %%%%%%%%%%%
\newcounter{rok}
\newcommand{\rokakademicki}{%
   \setcounter{rok}{\number\year}%
   \ifthenelse{\number\month<10}%
   {\addtocounter{rok}{-1}}% rok akademicki zaczal sie w pazdzierniku poprzedniego roku
   {}%                       rok akademicki zaczyna sie w pazdzierniku tego roku
   \arabic{rok}/\addtocounter{rok}{1}\arabic{rok}
}
%%%%%%%%%%%%%%%%%%%%%%%%%%%%%%%%%%%%%%%


%%%% LISTA UWAG %%%%%%%%%
\usepackage{color}
\definecolor{brickred}      {cmyk}{0   , 0.89, 0.94, 0.28}

\makeatletter \newcommand \kslistofremarks{\section*{Uwagi} \@starttoc{rks}}
\newcommand\l@uwagas[2]
{\par\noindent \textbf{#2:} %\parbox{10cm}
   {#1}\par} \makeatother


\newcommand{\ksremark}[1]{%
   {{\color{brickred}{[#1]}}}%
   \addcontentsline{rks}{uwagas}{\protect{#1}}%
}

\newcommand{\comma}{\ksremark{przecinek}}
\newcommand{\nocomma}{\ksremark{bez przecinka}}
\newcommand{\styl}{\ksremark{styl}}
\newcommand{\ortografia}{\ksremark{ortografia}}
\newcommand{\fleksja}{\ksremark{fleksja}}
\newcommand{\pauza}{\ksremark{pauza `--', nie dywiz `-'}}
\newcommand{\kolokwializm}{\ksremark{kolokwializm}}
\newcommand{\cytowanie}{\ksremark{cytowanie}}

%%%%%%%%%%%%%%%%%%%%%%%%%
%%%%%%%%%%%%%%%%%%%%%%%%%
%%%%%%%%%%%%%%%%%%%%%%%%%
%%%%%%%%%%%%%%%%%%%%%%%%%
%%%%%%%%%%%%%%%%%%%%%%%%%
%%%%%%%%%%%%%%%%%%%%%%%%%
%%%%%%%%%%%%%%%%%%%%%%%%%
%%%%%%%%%%%%%%%%%%%%%%%%%
%%%%%%%%%%%%%%%%%%%%%%%%%
%%%%%%%%%%%%%%%%%%%%%%%%%
%%%%%%%%%%%%%%%%%%%%%%%%%
%%%%%%%%%%%%%%%%%%%%%%%%%



% autor:
\fancyhead[RE]{\small\bfseries Daniel Wiszowaty} % autor sprawozdania



%%%%%%%%%%% NO I ZACZYNA SIE SPRAWOZDANIE %%%%%%%%%%%

\begin{document}
\frenchspacing
\thispagestyle{empty}
\begin{center}
{\Large\sf Politechnika Śląska   % Alma Mater

Wydział Informatyki, Elektroniki i Informatyki

}

\vfill

 

\vfill\vfill

{\Huge\sffamily\bfseries Programowanie Komputerów 3\par}  

\vfill\vfill

{\LARGE\sf Rezerwacje w kinie}   


\vfill \vfill\vfill\vfill

%%%%%%%%%%%%%%%%%%%%%%%%%%%%





\begin{tabular}{ll}
	\toprule
	autor                       & Daniel Wiszowaty    \\
	prowadzący                  & mgr. Grzegorz Wojciech Kwiatkowski  \\
	rok akademicki              & \rokakademicki         \\
	kierunek                    & informatyka            \\
	rodzaj studiów              & SSI                    \\
	semestr                     & 3                      \\
	sekcja                      & 21                     \\
	termin oddania sprawozdania & 2020-11-10             \\
	git & \href{https://github.com/polsl-aei-pk3/aa44cccc-gr22-repo}{polsl-aei-pk3 / aa44cccc-gr22-repo} \\
	\bottomrule
	                            &
\end{tabular}

\end{center}

%%%%%%%%%%%%%%%%%%%%%%%%%%%%%%%%%%%%%%%%%%%%%%%%%%%%%%%%%%%%%%%%%%%%%%%%%
\cleardoublepage
%%%%%%%%%%%%%%%%%%%%%%%%%%%%%%%%%%%%%%%%%%%%%%%%%%%%%%%%%%%%%%%%%%%%%%%%%

%%%%%%%%%%%%%%%%%%%%%%%%%%%%%%%%%%%%%%%%%%%%%%%%%%%%%%%%%%%%%%%%%%%%%%%%%
\section{Treść zadania}
\marginpar{}

Zadanie polega na utworzeniu symulacji \texttt{obsługi rezerwacji w kinie}. Program umożliwia dodanie seansu, rezerwację i zakup biletów, wypisanie rezerwacji a także wypisywanie ich w zależności z uwzględnieniem daty oraz numeru sali.


%%%%%%%%%%%%%%%%%%%%%%%%%%%%%%%%%%%%%%%%%%%%%%%%%%%%%%%%%%%%%%%%%%%%%%%%%
\section{Analiza zadania}
\marginpar{}

Zagadnienie przedstawia problem \texttt{quasi-bazy danych}. 

\subsection{Struktury danych}
\marginpar{}
W programie wykorzystano dwie listy jednokierunkowe. Pierwsza lista zawiera informacje o \texttt{sali}. Druga lista przechowuje informacje o rezerwacji tj. \texttt{dacie}, \texttt{sali}, \texttt{bilecie} i \texttt{nazwie filmu}. Taka struktura umożliwia sprawne zarządzanie rezerwacjami. \newline

\begin{figure}[H]
\centering
\begin{tikzpicture}

\node at (1,7) [rectangle,draw] (Kowalska) {$18$};
	\node at (5,7) [rectangle,draw] (Kamiński) {$26$};
		\node at (9,7) [rectangle,draw] (Nowak) {$6$};
		
\draw[>=latex,->] (Kowalska) -- (Kamiński);
\draw[>=latex,->] (Kamiński) -- (Nowak);

\end{tikzpicture}
\indent \caption{Przykład listy jednokierunkowej.}
\label{fig:lista_jednokierunkowa}
\end{figure} 



\noindent Uważam, że strukturę można usprawnić zmieniając ją w \texttt{listę podwieszaną}. W ten sposób każda sala będzie miała swoją listę rezerwacji, co znacząco usprawni działanie programu.

\begin{figure}[H]
\centering
\begin{tikzpicture}
            
%draw [help lines] grid (7,7);
\node at (1,7) [rectangle,draw] (Kowalska) {$A$};
	\node at (5,7) [rectangle,draw] (Kamiński) {$B$};
		\node at (9,7) [rectangle,draw] (Nowak) {$C$};
\node at (1, 5) [rectangle,draw] (A) {$15$};
	\node at (1, 3) [rectangle,draw] (B) {$12$};
			\node at (1, 1) [rectangle,draw] (C) {$17$};

\node at (5, 5) [rectangle,draw] (E) {$4$};
	\node at (5, 3) [rectangle,draw] (F) {$6$};
	
\node at (9, 5) [rectangle,draw] (G) {$13$};
		\node at (9, 3) [rectangle,draw] (H) {$28$};
			\node at (9, 1) [rectangle,draw] (I) {$18$};



\draw[>=latex,->] (Kowalska) -- (Kamiński);
\draw[>=latex,->] (Kamiński) -- (Nowak);
\draw[>=latex,->] (Kowalska) -- (A);
\draw[>=latex,->] (A) -- (B);
\draw[>=latex,->] (B) -- (C);
\draw[>=latex,->] (Kamiński) -- (E);
\draw[>=latex,->] (E) -- (F);
\draw[>=latex,->] (Nowak) -- (G);
\draw[>=latex,->] (G) -- (H);
\draw[>=latex,->] (H) -- (I);


\end{tikzpicture}
\caption{Przykład listy podwieszanej.}
\label{fig:lista_podwieszana}
\end{figure}


\subsection{Algorytmy}
\marginpar{}
Program dodaje rezerwacje do listy jednokierunkowej uwzględniając, żeby nie utworzyć dwóch sal o tej samej dacie i jej unikalnym numerze. Wypisywanie list odbywa się poprzez iteracyjne przejście, z przeładowaną funkcją, uwzględniającą datę i nazwę filmu. \cite{id:PPK}


%%%%%%%%%%%%%%%%%%%%%%%%%%%%%%%%%%%%%%%%%%%%%%%%%%%%%%%%%%%%%%%%%%%%%%%%%
\section{Specyfikacja zewnętrzna}
\marginpar{}

Program jest uruchamiany z konsoli.
Przy uruchomieniu programu pojawia się menu:
\begin{verbatim}
Witaj w programie do obslugi rezerwacji kin           
______________________________________________________

Obslugiwane komendy:                                  
instrukcja                                            
zapisz <nazwapliku>                                   
odczytaj <nazwapliku>                                   
wyjdz                                                 
dodajFilm <data> <sala> <nazwaFilmu>                  
dodajFilm <data> <sala> <nazwaFilmu>                  
zabookuj <rzad> <kolumna> <data> <nazwaFilmu>         
kup <rzad> <kolumna> <data> <nazwaFilmu>              
odbookuj <rzad> <kolumna> <data> <nazwaFilmu>         
wypiszSale                                                 
rezerwacje [<data>] [<nazwaFilmu>]             

\end{verbatim}

\begin{Verbatim}[fontsize=\small]
instrukcja
\end{Verbatim}
powoduje wyświetlenie instrukcji. \newline

\begin{Verbatim}[fontsize=\small]
zapisz <nazwapliku>
\end{Verbatim}
powoduje zapis całego programu do pliku przekazanego przez argument \texttt{nazwapliku}. \newline

\begin{Verbatim}[fontsize=\small]
odczytaj <nazwapliku>
\end{Verbatim}
powoduje odczyt z pliku przekazanego przez argument \texttt{nazwapliku}. \newline

\begin{Verbatim}[fontsize=\small]
wyjdz
\end{Verbatim}
powoduje koniec programu. \newline

\begin{Verbatim}[fontsize=\small]
dodajFilm <data> <sala> <nazwaFilmu>                  
\end{Verbatim}
powoduje dodanie nowego filmu z datą, salą i nazwą przekazaną przez argumenty \texttt{data}, \texttt{sala} i \texttt{nazwaFilmu}. \newline

\begin{Verbatim}[fontsize=\small]
usunFilm <data> <sala> <nazwaFilmu>                  
\end{Verbatim}
powoduje usunięcie filmu z datą, salą i nazwą przekazaną przez argumenty \texttt{data}, \texttt{sala} i \texttt{nazwaFilmu}. \newline

\begin{Verbatim}[fontsize=\small]
zabookuj <rzad> <kolumna> <data> <nazwaFilmu>         
\end{Verbatim}
powoduje zarezerwowanie biletu na film o nazwie \texttt{nazwaFilmu} na fotelu przekazanym przez argumenty \texttt{rzad} i \texttt{kolumna} z datą \texttt{data}. \newline

\begin{Verbatim}[fontsize=\small]
kup <rzad> <kolumna> <data> <nazwaFilmu>              
\end{Verbatim}
powoduje kupienie biletu na film o nazwie \texttt{nazwaFilmu} na fotelu przekazanym przez argumenty \texttt{rzad} i \texttt{kolumna} z datą \texttt{data}. \newline

\begin{Verbatim}[fontsize=\small]
odbookuj <rzad> <kolumna> <data> <nazwaFilmu>         
\end{Verbatim}
powoduje usunięcie rezerwacji na film o nazwie \texttt{nazwaFilmu} na fotelu przekazanym przez argumenty \texttt{rzad} i \texttt{kolumna} z datą \texttt{data}. \newline

\begin{Verbatim}[fontsize=\small]
wypiszSale                                                 
\end{Verbatim}
powoduje wypisanie na konsolę wszystkich sal filmowych. \newline

\begin{Verbatim}[fontsize=\small]
rezerwacje [<data>] [<nazwaFilmu>]                                                            
\end{Verbatim}
powoduje wypisanie na konsolę wszystkich rezerwacji z opcjonalnymi parametrami \texttt{data} i \texttt{nazwaFilmu}.


\subsection{Szczegółowy opis typów i funkcji}

Szczegółowy opis typów i funkcji zawarty jest w załączniku. 

\section{Testowanie}
\marginpar{}

Program został sprawdzony pod kątem wycieków pamięci przy użyciu biblioteki \texttt{nvwa} i \texttt{-fsanitize=leak} (biblioteki Clang).


\section{Wnioski}
\marginpar{}
Program do obsługi rezerwacji nie był programem trudnym, ale kluczowe było zastosowanie odpowiednich struktur.

 
\begin{filecontents}{bibliografia.bib}
@misc{id:PPK,
title = {Wykłady z podstaw programowania komputer\'ow},
author = {Krzysztof Simiński},
}

\end{filecontents}


\bibliographystyle{plplain}
\bibliography{bibliografia}

 
\cleardoublepage

\rule{0cm}{0cm}

\vfill

\begin{center}
\Huge\bfseries Dodatek\\Szczegółowy opis typów i~funkcji\par
\end{center}

\vfill 

\rule{0cm}{0cm}

\end{document}
% Koniec wieńczy dzieło.
